\documentclass{amsart}
\title{A \LaTeX\ Starter File}
\author{Edderic Ugaddan}
\address{Box 666\\University of Washington\\Seattle\ \ WA}
\email{edderic@gmail.com}
\date{Nov. 05, 2016}

% Check if we are compiling under latex or pdflatex, and include the
% appropriate graphics package
\ifx\pdftexversion\undefined
  \usepackage[dvips]{graphicx}
\else
  \usepackage[pdftex]{graphicx}
\fi

% Define the theorem environments
\newtheorem{theorem}{Theorem}[section] % numbered like the section
\newtheorem{lemma}[theorem]{Lemma} % numbered like the theorems
\newtheorem{proposition}[theorem]{Proposition}
\newtheorem{corollary}[theorem]{Corollary}
\theoremstyle{definition} % styled differently... not italicized
\newtheorem{definition}[theorem]{Definition}
\newtheorem{conjecture}[theorem]{Conjecture}

% Let \sign show roman-style characters in math mode
\newcommand{\sign}{\mathop{\mathrm{sign}}}

%\raggedbottom % Makes the bottom margin more flexible (helpful for pictures)

\begin{document}

\begin{abstract}
 Put an abstract here when you're ready.
\end{abstract}

\maketitle

\tableofcontents

\section{Introduction}

Introduction.  We might cite bibliographic entry \cite{SJE} here.

\section{Section One}

Section One.  This is the section that has an in-text equation
\(e=mc^2\) and a centered equation or two:

\begin{equation}
  P(X_{1} = 2 \text{ and } X_{2} = 1) = P(X_{2} = 1 \mid X_{1} = 2) P(X_{1} = 2)
\end{equation}

Formally speaking:

\begin{equation}
\begin{split}
  P(X_{1} = 2) &= \sum_{i=0}^{2}{P(X_{1} = 2 \mid X_{0} = i) P(X_{0} = i)} \\
  &= P(X_{1} = 2 \mid X_{0} = 0) P(X_{0} = 0) \\
  &\quad +  P(X_{1} = 2 \mid X_{0} = 1) P(X_{0} = 1) \\
  &\quad + P(X_{1} = 2 \mid X_{0} = 2) P(X_{0} = 2)
\end{split}
\end{equation}


However, since it was given that both nodes were active in the beginning (i.e.
$P(X_{0} = 2) = 1$), we know that $P(X_{0} = 0) = P(X_{0} = 1) = 0$. Thus, the
statement above simplifies to:

\begin{equation}
\begin{aligned}
  P(X_{1} = 2) &= P(X_{1} = 2 \mid X_{0} = 2)
\end{aligned}
\end{equation}

$P(X_{1} = 2 \mid X_{0} = 2)$ could only happen two ways: either both
active nodes send information (and hence create a collision), or both active
nodes don't send anything at all.

\begin{equation}
  \begin{aligned}
    P(X_{1} = 2) &= P(X_{1} = 2 \mid X_{0} = 2) \\
    &= P(C_{1} = 1 \text{ and } C_{2} = 1 \text{ or } C_{1} = 0 \text{ and } C_{2} = 0) \\
                 &= P(C_{1} = 1 \text{ and } C_{2} = 1) + P(C_{1} = 0 \text{ and } C_{2} = 0) \\
    &= p^2 + (1-p)^2 \\
    &= (0.4)^2 + (1-0.4)^2 \\
    &= 0.16 + 0.36 \\
    &= 0.52 \\
  \end{aligned}
\end{equation}

\begin{equation}
  \begin{aligned}
    P(X_{2} = 1 \mid X_{1} = 2) &= P(C_{1} = 1 \text{ and } C_{2} = 0 \text{ or } C_{1} = 0 \text{ and } C_{2} = 1) \\
    &= P(C_{1} = 1 \text{ and } C_{2} = 0) + P(C_{1} = 0 \text{ and } C_{2} =  1) \\
    &= p(1-p) + p(1-p) \\
    &= 2p(1-p) \\
    &= 2(0.4)(1-(0.4)) \\
    &= 0.48\\
  \end{aligned}
\end{equation}

Therefore:

\begin{equation}
  \begin{aligned}
    P(X_{1} = 2 \text{ and } X_{2} = 1) &= P(X_{2} = 1 \mid X_{1} = 2) P(X_{1} = 2) \\
    &= 0.48 \times 0.52 \\
    &= 0.2496
  \end{aligned}
\end{equation}


\begin{equation}
  \begin{aligned}
    P(X_2 = 0) &= \sum_{i=0}^{2} P(X_2 = 0 \text{ and } X_1 = i) \\
    &= \sum_{i=0}^{2} P(X_2 = 0 \mid X_1 = i)P(X_1 = i) \\
    &= P(X_2 = 0 \mid X_1 = 0)P(X_1 = 0) \\
    &\quad + P(X_2 = 0 \mid X_1 = 1)P(X_1 = 1) \\
    &\quad + P(X_2 = 0 \mid X_1 = 2)P(X_1 = 2) \\
  \end{aligned}
\end{equation}

Let's start with $P(X_1 = 0)$. We know that it is equivalent to:

\begin{equation}
  \begin{aligned}
    P(X_1 = 0) &= P(X_1 = 0 \text{ and } X_0 = 2) \\
    &= P(X_1 = 0 \mid X_0 = 2)P(X_0 = 2) \\
    &= P(X_1 = 0 \mid X_0 = 2) \\
  \end{aligned}
\end{equation}

It is impossible for two active nodes to both become inactive for the next
epoch, so $P(X_1 = 0) = P(X_1 = 0 \mid X_0 = 2) = 0$.

Next we look at $P(X_1 = 1)$:

\begin{equation}
  \begin{aligned}
    P(X_1 = 1) &= P(X_1 = 1 \text{ and } X_0 = 2) \\
    &= P(X_1 = 1 \mid X_0 = 2)P(X_0 = 2) \\
    &= P(X_1 = 1 \mid X_0 = 2) \\
  \end{aligned}
\end{equation}

This could only happen two ways: either the first node sends and the second
node does not, or the second node sends and the first node does not. For active
nodes, let $S_i=k, i \in \{0,1\}$ and $k \in \{0,1\}$ where, for node $i$,
$S_i=0$ is the event of not sending and $S_i=1$ is the event of sending:

\begin{equation}
  \begin{aligned}
    P(X_1 = 1) &= P(X_1 = 1 \mid X_0 = 2) \\
    &= P(S_1 = 1 \text{ and } S_2 = 0 \text{ or } S_1 = 0 \text{ and } S_2 = 1) \\
    &= P(S_1 = 1 \text{ and } S_2 = 0) + P(S_1 = 0 \text{ and } S_2 = 1) \\
    &= p(1-p) + p(1-p) \\
    &= 2p(1-p) \\
    &= 2(0.4)(1-0.4) \\
    &= 0.48
  \end{aligned}
\end{equation}

Now let's look at $P(X_2=0 \mid X_1=1)$. The only way this could happen is when
the active node sends while the other node stays inactive:

\begin{equation}
  \begin{aligned}
    P(X_2=0 \mid X_1=1) &= p(1-q) \\
    &= (0.4)(1-(0.8)) \\
    &= 0.08
  \end{aligned}
\end{equation}

Now we consider $P(X_2=0 \mid X_1=2)$. Again it is impossible for two active
nodes to become inactive for the next epoch, so $P(X_2=0 \mid X_1=2) = 0$.

Therefore, $P(X_2=0)$ simplifies to the following:

\begin{equation}
  \begin{aligned}
    P(X_2=0) &= P(X_2 = 0 \mid X_1 = 0)P(X_1 = 0) \\
    &\quad + P(X_2 = 0 \mid X_1 = 1)P(X_1 = 1) \\
    &\quad + P(X_2 = 0 \mid X_1 = 2)P(X_1 = 2) \\
    &= P(X_2 = 0 \mid X_1 = 0) \times 0 \\
    &\quad + 0.08 \times 0.48 \\
    &\quad + 0 \times P(X_1 = 2) \\
    &= 0.0384
  \end{aligned}
\end{equation}

Find $P(X_{1} = 1 \mid X_{2} = 1)$


\begin{equation}
  \begin{aligned}
    P(X_{1} = 1 \mid X_{2} = 1) &= \frac{P(X_1=1 \text{ and } X_2=1)}{P(X_2 = 1)} \\
    &= \frac{P(X_2=1 \mid X_1=1)P(X_1=1) }{P(X_2 = 1)} \\
    &= \frac{P(X_2=1 \mid X_1=1)P(X_1=1) }{\sum_{i=0}^{2}P(X_2 = 1, X_1=i)} \\
    &= \frac{P(X_2=1 \mid X_1=1)P(X_1=1) }{\sum_{i=0}^{2}P(X_2 = 1 \mid X_1=i)P(X_1=i)} \\
  \end{aligned}
\end{equation}

The denominator expands to the following:

\begin{equation}
  \begin{aligned}
    \sum_{i=0}^{2}P(X_2 = 1 \mid X_1=i)P(X_1=i) &= P(X_2 = 1 \mid X_1=0)P(X_1=0) \\
    &\quad + P(X_2 = 1 \mid X_1=1)P(X_1=1) \\
    &\quad + P(X_2 = 1 \mid X_1=2)P(X_1=2) \\
  \end{aligned}
\end{equation}

We've already established that $P(X_1=0) = 0$, $P(X_1=1)=0.48$, $P(X_1=2) =
0.52$. Since, $P(X_1=0) = 0$, we know that $P(X_2=1 \mid X_1=0)$ cannot
happen. Thus, what we need to calculate are the following: $P(X_2=1
\mid X_1=1)$ and $P(X_2=1 \mid X_1=2)$.

Let's start with $P(X_2=1 \mid X_1=1)$. Three possibilities:

\begin{itemize}
 \item Active node successfully sends; inactive node becomes active and does not send
 \item Active node does not attempt to send; inactive node does not become active
 \item Active node does not attempt to send; inactive node becomes active and successfully sends
\end{itemize}

\begin{equation}
  \begin{aligned}
    P(X_2=1 \mid X_1=1) &= (1-p)qp + (1-p)(1-q) + pq(1-p) \\
    &= (1-0.4)(0.8)(0.4) + (1-0.4)(1-0.8) + (0.4(0.8(1-0.4))) \\
    &= 0.192 + 0.12 + 0.192 \\
    &= 0.504
  \end{aligned}
\end{equation}

Next we look at $P(X_2=1 \mid X_1=2)$. The first active node sends a message
and becomes inactive while the second active node refrains from sending, or the
second active node sends a message and becomes inactive while the first active
node refrains.

\begin{equation}
  \begin{aligned}
    P(X_2=1 \mid X_1=2) &= P(A_1=1 \text{ and } A_2=0 \text{ or } A_1=0 \text{ and } A_2=1) \\
    &= P(A_1=1 \text{ and } A_2=0) + P(A_1=0 \text{ and } A_2=1) \\
    &= p(1-p) + (1-p)p \\
    &= 2p(1-p) \\
    &= 2(0.4)(1-0.4) \\
    &= 0.48
  \end{aligned}
\end{equation}

Therefore, $P(X_1=1 \mid X_2=1)$ reduces to the following:

\begin{equation}
  \begin{aligned}
    P(X_1=1 \mid X_2=1) &= \frac{P(X_2=1 \mid X_1=1)P(X_1=1) }{\sum_{i=0}^{2}P(X_2 = 1 \mid X_1=i)P(X_1=i)} \\
    &= \frac{0.504 \times 0.48}{0 \times 0 + 0.504 \times 0.48 + 0.48 \times 0.52}\\
    &= \frac{0.24192}{0 + 0.24192 + 0.2496} \\
    &= 0.4921875 \\
\end{equation}

\begin{equation}
 \int_{-\infty}^\infty e^{-t^2}dt = \sqrt{\pi}
\end{equation}

\begin{equation*} % The star suppresses the equation numbering:
 \sum_1^\infty \frac{1}{n^2} = \frac{\pi^2}{6}
\end{equation*}

And another equation using a matrix.

\begin{equation}
 \textrm{Matrix to follow\ldots }
 \Lambda =
 \begin{bmatrix}
   3        &  \nabla \\
   \partial &  b
 \end{bmatrix}
\end{equation}

Greek letters are easy to use:

\begin{equation}
 \alpha \times \delta \rightarrow \frac{\mathbb R}{\phi}
\end{equation}

\subsection{Section One Point One}

Section One Point One.  This is the section that has the figure.  It
might appear on the next page.  One can then reference it as Figure
\ref{a figure} on page \pageref{a figure}.

\begin{figure}
\begin{center}
% Leave off the file extension so that
% the package can choose from .eps and .pdf
%\includegraphics[width=3in]{circle}
% Or if you want to waste your time, make a LaTeX figure:
\begin{picture}(50,50)
\put(25,25){\circle{50}}
\end{picture}
\end{center}
\caption{This is a big circle.  \label{a figure}}
\end{figure}

\subsection{Section One Point Two}

Section One Point Two.  This is the section that has a bulleted list.
You can number the list using \texttt{enumerate} instead of
\texttt{itemize}.

\begin{itemize}
 \item Item 1
 \item Item 2
\end{itemize}

You can also look up \texttt{description} if you're interested.

\section{I forget which number}

This is the section which has a code sample:

\begin{verbatim}
     ((lambda (x)
       (list x (list (quote quote) x)))
      (quote
         (lambda (x)
           (list x (list (quote quote) x)))))
\end{verbatim}


\section{Conclusion}

Conclusion goes here.

\appendix

\section{Stuff That Doesn't Go Up There Goes Here Instead}

This item goes without notice 19 years out of 20, but occaisionally
must be surgically removed.


\begin{thebibliography}{99}

\bibitem{SJE}Sam, Jeff, Erine.  ``A \LaTeX\ Starter File.''  2003.

\bibitem{CM}Burtis, B., and James A. Morrow.  ``Inverse Problems
for Electrical Networks.''  Series on applied mathematics -- Vol.\ 13.
World Scientific, \copyright 2000.

\end{thebibliography}

\end{document}

